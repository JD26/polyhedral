%% LyX 2.3.6 created this file.  For more info, see http://www.lyx.org/.
%% Do not edit unless you really know what you are doing.
\documentclass[english]{article}
\usepackage[T1]{fontenc}
\usepackage[latin9]{inputenc}
\setlength{\parindent}{0bp}
\usepackage{color}
\definecolor{note_fontcolor}{rgb}{0.800781, 0.800781, 0.800781}
\usepackage{babel}
\usepackage{textcomp}
\usepackage{mathtools}
\usepackage{amsmath}
\usepackage{amsthm}
\usepackage{amssymb}
\usepackage[unicode=true]
 {hyperref}

\makeatletter

%%%%%%%%%%%%%%%%%%%%%%%%%%%%%% LyX specific LaTeX commands.
%% The greyedout annotation environment
\newenvironment{lyxgreyedout}
  {\textcolor{note_fontcolor}\bgroup\ignorespaces}
  {\ignorespacesafterend\egroup}

%%%%%%%%%%%%%%%%%%%%%%%%%%%%%% Textclass specific LaTeX commands.
\theoremstyle{plain}
\newtheorem{thm}{\protect\theoremname}[section]
\theoremstyle{definition}
\newtheorem{defn}[thm]{\protect\definitionname}
\theoremstyle{plain}
\newtheorem{prop}[thm]{\protect\propositionname}
\ifx\proof\undefined
\newenvironment{proof}[1][\protect\proofname]{\par
	\normalfont\topsep6\p@\@plus6\p@\relax
	\trivlist
	\itemindent\parindent
	\item[\hskip\labelsep\scshape #1]\ignorespaces
}{%
	\endtrivlist\@endpefalse
}
\providecommand{\proofname}{Proof}
\fi
\theoremstyle{remark}
\newtheorem{rem}[thm]{\protect\remarkname}

%%%%%%%%%%%%%%%%%%%%%%%%%%%%%% User specified LaTeX commands.
\usepackage{crossreftools}

\makeatother

\providecommand{\definitionname}{Definition}
\providecommand{\propositionname}{Proposition}
\providecommand{\remarkname}{Remark}
\providecommand{\theoremname}{Theorem}

\begin{document}

\section{Notation}

The zero vector of $\mathbb{R}^{d}$ is denoted by $\vec{0}$.

The unit sphere $\left\{ x\in\mathbb{R}^{d}\mid\left\Vert x\right\Vert =1\right\} $
is denoted by $S^{d-1}$.

Given a finite set of indices $\mathcal{K}$, $\left|\mathcal{K}\right|$
denotes the cardinality of $\mathcal{K}$.

\section{Polyhedrals and polytopes}

Let us start with some definitions and facts. Our reference is \cite{ABrIP1983}.
\begin{defn}
An \emph{affine} \emph{subspace} of $\mathbb{R}^{d}$ is either the
empty set or a translate of a linear subspace, that is, a subset $x+L$
where $x\in\mathbb{R}^{d}$ and $L$ is a linear subspace of $\mathbb{R}^{d}$.
\end{defn}
%
\begin{defn}
The dimension of a non-empty affine subspace $A$ is the dimension
of the linear subspace $L$ such that $A=x+L$. In other words, 
\[
\mathrm{dim}\,A\coloneqq\mathrm{dim}\,L.
\]
\end{defn}
%
\begin{defn}
For any subset $M$ of $\mathbb{R}^{d}$, there is a smallest affine
subspace containing $M$, namely, the intersection of all affine subspaces
containing $M$. This affine subspace is called the \emph{affine hull}
of M, and it is denoted by $\mathrm{aff}\,$M.
\end{defn}
%
\begin{defn}
Let $C$ be a convex set of $\mathbb{R}^{d}$. The dimension of $C$
is defined as the dimension of the affine hull of $C$, that is 
\[
\mathrm{dim}\,C\coloneqq\mathrm{dim}\left(\mathrm{aff}\,C\right).
\]
\end{defn}
%
\begin{defn}
For any subset $M$ of $\mathbb{R}^{d}$, there is a smallest convex
set containing $M$, namely, the intersection of all convex sets containing
$M$. This convex set is called the \emph{convex hull} of M, and it
is denoted by $\mathrm{conv}\,$M. 
\end{defn}
%
\begin{lyxgreyedout}
>\textcompwordmark > The following is a definition adapted to this
context.%
\end{lyxgreyedout}

\begin{defn}
Let $C$ be a closed convex set of $\mathbb{R}^{d}$. A point $x\in C$
is called an \emph{extreme point} of $C$ if $C\backslash\left\{ x\right\} $
is again convex. The set of extreme points of $C$ is denoted by $\mathrm{ext}\,C$.
\end{defn}
%
\begin{thm}[Minkowski's Theorem]
Let $C$ be a compact set in $\mathbb{R}^{d}$. Then
\[
C=\mathrm{conv}\left(\mathrm{ext}\,C\right).
\]
\end{thm}
%
\begin{defn}
A subset $P$ of $\mathbb{R}^{d}$ is called \emph{polyhedral set}
if $Q$ is the intersection of a finite number of closed half-spaces
or $P=\mathbb{R}^{d}$. Every polyhedral set is closed and convex.
A polyhedral set $P\neq\mathbb{R}^{d}$ has the form
\begin{equation}
P=\bigcap_{k\in\mathcal{K}}\left\{ x\in\mathbb{R}^{d}\mid x\cdot n_{k}\leq a_{k}\right\} ,\label{eq:rep}
\end{equation}
where $\mathcal{K}$ is a finite set of indices, $n_{k}\in\mathbb{R}^{d}\backslash\left\{ \vec{0}\right\} $
and $a_{k}\in\mathbb{R}$. In (\ref{eq:rep}), it is implicitly assumed
that no two closed half-spaces are identical.
\end{defn}
%
\begin{defn}
The representation (\ref{eq:rep}) of $P$ is called \emph{irreducible}
if $\left|\mathcal{K}\right|=1$, or $\left|\mathcal{K}\right|>1$
and
\[
P\subsetneq\bigcap_{k\in\mathcal{K}\backslash\left\{ j\right\} }\left\{ x\in\mathbb{R}^{d}\mid x\cdot n_{k}\leq a_{k}\right\} \quad\forall\;j\in\mathcal{K}.
\]
A representation which is not irreducible is called \emph{reducible}.
\end{defn}
%
\begin{thm}
Let $P$ be a polyhedral set in $\mathbb{R}^{d}$ with $\mathrm{dim}\,P=d$
and $P\neq\mathbb{R}^{d}$. Suppose that $\left|\mathcal{K}\right|>1$
in the representation (\ref{eq:rep}) of $P$. Then (\ref{eq:rep})
is irreducible if and only if
\[
\left\{ x\in\mathbb{R}^{d}\mid x\cdot n_{j}=a_{j}\right\} \cap\bigcap_{k\in\mathcal{K}\backslash\left\{ j\right\} }\left\{ x\in\mathbb{R}^{d}\mid x\cdot n_{k}\leq a_{k}\right\} \neq\emptyset\quad\forall\;j\in\mathcal{K}.
\]
\end{thm}
%
In this case, each set $\left\{ x\in\mathbb{R}^{d}\left|\,x\cdot n_{j}=a_{j}\right.\right\} \cap Q$
is called face of $Q$. It turns that every face of $Q$ is also a
polyehedral set and the number of faces of $Q$ is finite.
\begin{defn}
A polytope is the convex hull of a non-empty finite set of $\mathbb{R}^{d}$. 
\end{defn}
%
A subset $P$ of $\mathbb{R}^{d}$ is a polytope if and only if it
is a non-empty bounded polyehedral set. A subset $P$ of $\mathbb{R}^{d}$
is a polytope if and only if it is a compact convex set with a finite
number of extreme points. Therefore, a non-empty bounded polyhedral
set $P$ is a compact convex set that can be represented as the convex
hull of its finite set of extreme points. These extreme points are
called vertices of $P$. A polytope of dimension $d$ is simple if
each vertex of $P$ is contained in precisely $d$ faces (reescribir).

\section{Barycentric coordinates}

Let $P$ be a non-empty bounded polyehedral set, or equivalently a
polytope, of dimension $d$. Assume that $P$ is simple. Let $\mathcal{V}$
be the set of vertices $P$. From \cite{JWaBP1996}\href{file:references/JWaBP1996.pdf}{�}\cite{JWaBS2007}\href{file:references/JWaBS2007.pdf}{�},
there exist a set of smooth functions $b_{v}:P\rightarrow\mathbb{R}$,
with $v\in\mathcal{V}$, called barycentric coordinates with respect
to $P$, satisfying
\begin{equation}
b_{v}\left(x\right)\geq0\quad\forall\,x\in P\:\mathrm{and}\:\forall\,v\in\mathcal{V},\label{eq:bc1}
\end{equation}
\begin{equation}
\sum_{v\in\mathcal{V}}b_{v}\left(x\right)=1\quad\forall\,x\in P,\label{eq:bc02}
\end{equation}
\begin{equation}
\sum_{v\in\mathcal{V}}b_{v}\left(x\right)v=x\quad\forall\,x\in P.\label{eq:bc03}
\end{equation}
Some properties of these functions are the following.
\begin{enumerate}
\item Let $l$ be an affine function. Then $l\left(x\right)=\sum_{v\in\mathcal{V}}b_{v}\left(x\right)l\left(v\right)$
for all $x\in P$.
\item Let $v\in\mathcal{V}$ be a vertex of $P$. Then $b_{v}\left(v\right)=1$
and $b_{v}\left(w\right)=0$ $\forall\,w\in\mathcal{V}\backslash\left\{ v\right\} $.
Moreover, $\forall$ $x\in P\backslash\left\{ v\right\} $ $\exists$
$w\in\mathcal{V}\backslash\left\{ v\right\} $ such that $b_{w}\left(x\right)>0$.
\begin{enumerate}
\item[] \emph{Proof.} Let $l:\mathbb{R}^{3}\rightarrow\mathbb{R}$ be affine
function such that $l\left(v\right)=0$ and $l\left(w\right)<0$ for
all $w\in\mathcal{V}\backslash\left\{ v\right\} $. Thus, $l^{-1}\left(\left\{ 0\right\} \right)$
defines a plane that touches $P$ only at $v$. We have
\[
0=l\left(v\right)=\sum_{w\in\mathcal{V}}b_{w}\left(v\right)l\left(w\right)=\sum_{w\in\mathcal{V}\backslash\left\{ v\right\} }b_{w}\left(v\right)l\left(w\right)
\]
because $v\in l^{-1}\left(\left\{ 0\right\} \right)$. Now each $b_{w}\left(v\right)\geq0$
and $l\left(w\right)<0$. Therfore $b_{w}\left(v\right)=0$ for all
$w\in\mathcal{V}\backslash\left\{ v\right\} $. Since the coordinate
functions sum $1$, $b_{v}\left(v\right)=1$. Let $x\in P\backslash\left\{ v\right\} $.
Then
\[
0>l\left(x\right)=\sum_{w\in\mathcal{V}\backslash\left\{ v\right\} }b_{w}\left(x\right)l\left(w\right).
\]
If $b_{w}\left(x\right)=0$ for all $w\in\mathcal{V}\backslash\left\{ v\right\} $,
then $l\left(x\right)=0$.
\end{enumerate}
\item Let $e$ be a edge of $P$ and $v,w$ be its vertices ($\overline{e}=e\cup\left\{ v,w\right\} $).
Then $b_{v}\left(x\right)+b_{w}\left(x\right)=1$ and $b_{u}\left(x\right)=0$
for all $u\in\mathcal{V}\backslash\left\{ v,w\right\} $, for all
$x\in e$. Moreover, for all $x\in P\backslash\overline{e}$ there
exists at least one vertex $u\in\mathcal{V}\backslash\left\{ v,w\right\} $
such that $b_{u}\left(x\right)>0$. Also
\[
b_{v}\left(x\right)=\frac{\left\langle x-w,v-w\right\rangle }{\left|\overline{e}\right|^{2}},\quad b_{w}\left(x\right)=\frac{\left\langle x-v,w-v\right\rangle }{\left|\overline{e}\right|^{2}}.
\]
Therefore, $b_{v}\left(x\right),b_{w}\left(x\right)>0$ for all $x\in e$.
\begin{enumerate}
\item[] \emph{Proof.} Let $l:\mathbb{R}^{3}\rightarrow\mathbb{R}$ be affine
function such that $l\left(\overline{e}\right)=0$ and $l\left(u\right)<0$
for all $u\in\mathcal{V}\backslash\left\{ v,w\right\} $. Thus, $l^{-1}\left(\left\{ 0\right\} \right)$
defines a plane that touches $P$ only at $\overline{e}$. We have
\[
0=l\left(x\right)=\sum_{u\in\mathcal{V}}b_{u}\left(x\right)l\left(u\right)=\sum_{u\in\mathcal{V}\backslash\left\{ v,w\right\} }b_{w}\left(x\right)l\left(u\right)\quad\forall\,x\in e
\]
because $\left\{ v,w\right\} \subset\overline{e}\subset l^{-1}\left(\left\{ 0\right\} \right)$.
Now each $b_{u}\left(x\right)\geq0$ and $l\left(u\right)<0$. Therfore
$b_{u}\left(x\right)=0$ for all $u\in\mathcal{V}\backslash\left\{ v,w\right\} $.
Since the coordinate functions sum $1$, $b_{v}\left(x\right)+b_{w}\left(x\right)=1$.
Let $x\in P\backslash\overline{e}$. Then
\[
0>l\left(x\right)=\sum_{u\in\mathcal{V}\backslash\left\{ v,w\right\} }b_{u}\left(x\right)l\left(u\right).
\]
If $b_{u}\left(x\right)=0$ for all $u\in\mathcal{V}\backslash\left\{ v,w\right\} $,
then $l\left(x\right)=0$. Note that
\[
x-v=\frac{\left\Vert x-v\right\Vert }{\left|\overline{e}\right|}\left(w-v\right),\quad x-w=\frac{\left\Vert x-w\right\Vert }{\left|\overline{e}\right|}\left(v-w\right)\qquad\forall\,x\in e.
\]
Since $x=b_{v}\left(x\right)v+b_{w}\left(x\right)w$ for all $x\in e$,
\[
x=b_{v}\left(x\right)v+\left(1-b_{v}\left(x\right)\right)w,\quad x=\left(1-b_{w}\left(x\right)\right)v+b_{w}\left(x\right)w
\]
and then
\[
\frac{\left\Vert x-w\right\Vert }{\left|\overline{e}\right|}\left(v-w\right)=b_{v}\left(x\right)\left(v-w\right),\quad\frac{\left\Vert x-v\right\Vert }{\left|\overline{e}\right|}\left(w-v\right)=b_{w}\left(x\right)\left(w-v\right).
\]
From this we deduce that
\[
b_{v}\left(x\right)=\frac{\left\Vert x-w\right\Vert }{\left|\overline{e}\right|}=\frac{\left\Vert x-w\right\Vert \left\Vert v-w\right\Vert }{\left|\overline{e}\right|^{2}}=\frac{\left|\left\langle x-w,v-w\right\rangle \right|}{\left|\overline{e}\right|^{2}}=\frac{\left\langle x-w,v-w\right\rangle }{\left|\overline{e}\right|^{2}}
\]
and
\[
b_{w}\left(x\right)=\frac{\left\Vert x-v\right\Vert }{\left|\overline{e}\right|}=\frac{\left\Vert x-v\right\Vert \left\Vert w-v\right\Vert }{\left|\overline{e}\right|^{2}}=\frac{\left|\left\langle x-v,w-v\right\rangle \right|}{\left|\overline{e}\right|^{2}}=\frac{\left\langle x-v,w-v\right\rangle }{\left|\overline{e}\right|^{2}}.
\]
\end{enumerate}
\item Let $f$ be a face of $P$ and $\mathcal{W}$ be the set of its vertices
($\overline{f}=f\cup\mathcal{W}\cup\left\{ \mathrm{edges}\right\} $).
Then $\sum_{w\in\mathcal{W}}b_{w}\left(x\right)=1$ and $b_{v}\left(x\right)=0$
for all $v\in\mathcal{V}\backslash\mathcal{W}$, for all $x\in f$.
Moreover, for all $x\in P\backslash\overline{f}$ there exists at
least one vertex $v\in\mathcal{V}\backslash\mathcal{W}$ such that
$b_{v}\left(x\right)>0$.
\begin{enumerate}
\item[] \emph{Proof.} Let $l:\mathbb{R}^{3}\rightarrow\mathbb{R}$ be affine
function such that $l\left(\overline{f}\right)=0$ and $l\left(v\right)<0$
for all $v\in\mathcal{V}\backslash\mathcal{W}$. Thus, $l^{-1}\left(\left\{ 0\right\} \right)$
defines a plane that touches $P$ only at $\overline{f}$. We have
\[
0=l\left(x\right)=\sum_{v\in\mathcal{V}}b_{v}\left(x\right)l\left(v\right)=\sum_{v\in\mathcal{V}\backslash\mathcal{W}}b_{v}\left(x\right)l\left(v\right)\quad\forall\,x\in f
\]
because $\mathcal{W}\subset\overline{f}\subset l^{-1}\left(\left\{ 0\right\} \right)$.
Now each $b_{v}\left(x\right)\geq0$ and $l\left(v\right)<0$. Therfore
$b_{v}\left(x\right)=0$ for all $v\in\mathcal{V}\backslash\mathcal{W}$.
Since the coordinate functions sum $1$, $\sum_{w\in\mathcal{W}}b_{w}\left(x\right)=1$.
Let $x\in P\backslash\overline{f}$. Then
\[
0>l\left(x\right)=\sum_{v\in\mathcal{V}\backslash\mathcal{W}}b_{v}\left(x\right)l\left(v\right).
\]
If $b_{v}\left(x\right)=0$ for all $v\in\mathcal{V}\backslash\mathcal{W}$,
then $l\left(x\right)=0$. 
\end{enumerate}
\item The coordinate functions $\left\{ b_{v}\right\} _{v\in\mathcal{V}}$
are linealy independent.
\begin{enumerate}
\item[] \emph{Proof.} Let $\left\{ \alpha_{v}\right\} _{v\in\mathcal{V}}$
be real numbers such that
\[
\sum_{v\in\mathcal{V}}\alpha_{v}b_{v}\left(x\right)=0\quad\forall\,x\in P.
\]
Setting $x=w\in\mathcal{V}$ it follows that $\alpha_{w}=\alpha_{w}b_{w}\left(w\right)=0$
(see property 1.).
\end{enumerate}
\end{enumerate}
%

\section{Polyhedral domain and deformation function}

Let $\mathcal{K}$ be a finite set of indices.
\begin{defn}
For each $k\in\mathcal{K}$, let $\widehat{\phi}_{k}:\mathbb{R}^{d}\times\mathbb{R}\rightarrow\mathbb{R}$
be the function defined by
\[
\widehat{\phi}_{k}\left(x,t\right)\coloneqq x\cdot n_{k}\left(t\right)-a_{k}\left(t\right),
\]
where $n_{k}:\mathbb{R}\rightarrow S^{d-1}$ and $a_{k}:\mathbb{R}\rightarrow\mathbb{R}$
are $C^{\infty}$ functions. Denote $n_{k}\coloneqq n_{k}\left(0\right)$,
$a_{k}\coloneqq a_{k}\left(0\right)$ and $\widehat{\phi}_{k}\left(\cdot\right)\coloneqq\widehat{\phi}_{k}\left(\cdot,0\right)$.
\end{defn}
%
\begin{defn}
For each $k\in\mathcal{K}$ and $t\in\mathbb{R}$, let $\omega_{k}\left(t\right)$
be the affine set defined by
\[
\omega_{k}\left(t\right)\coloneqq\left\{ x\in\mathbb{R}^{d}\mid\widehat{\phi}_{k}\left(x,t\right)<0\right\} .
\]
Denote $\omega_{k}\coloneqq\omega_{k}\left(0\right)$. Observe that
$\omega_{k}\left(t\right)$ is one of the two open half-spaces determined
by the hyperplane 
\[
\left\{ x\in\mathbb{R}^{d}\mid x\cdot n_{k}\left(t\right)=a_{k}\left(t\right)\right\} .
\]
\end{defn}
%
\begin{defn}
For each $t\in\mathbb{R}$, let $\Omega_{t}$ be the convex set of
$\mathbb{R}^{d}$ defined by
\[
\Omega_{t}\coloneqq\bigcap_{k\in\mathcal{K}}\omega_{k}\left(t\right).
\]
The set $\Omega_{t}$ is open because it is the finite intersection
of open half-spaces. Denote 
\[
\Omega\coloneqq\Omega_{0}=\bigcap_{k\in\mathcal{K}}\omega_{k}.
\]
As 
\[
\overline{\Omega}=\bigcap_{k\in\mathcal{K}}\overline{\omega_{k}}=\bigcap_{k\in\mathcal{K}}\left\{ x\in\mathbb{R}^{d}\mid x\cdot n_{k}\leq a_{k}\right\} 
\]
is a finite intersection of closed half-spaces, $\overline{\Omega}$
is a polyhedral set by definition.
\end{defn}
%
Consider the following four assumptions:
\begin{enumerate}
\item[{\crtcrossreflabel{$\mathbf{A1}$}[enu:A1].}] $a_{k}>0$ for all $k\in\mathcal{K}$.
\end{enumerate}
Assumption \ref{enu:A1} implies that $\overrightarrow{0}\in\Omega$.
Indeed
\[
\widehat{\phi}_{k}\left(\overrightarrow{0},0\right)\coloneqq\overrightarrow{0}\cdot n_{k}-a_{k}=-a_{k}<0,
\]
and hence
\[
\vec{0}\in\omega_{k}=\left\{ x\in\mathbb{R}^{d}\mid\widehat{\phi}_{k}\left(x,0\right)<0\right\} \quad\forall\;k\in\mathcal{K}.
\]
Moreover, one can deduce that $\dim\,\overline{\Omega}=d$. Indeed,
since $\overrightarrow{0}$ is a interior point of $\overline{\Omega}$,
there exists a point $y\in\overline{\Omega}\backslash\left\{ \vec{0}\right\} $
in the line segment joining $\vec{0}$ and any point $x\in\mathbb{R}^{d}$.
Thus $x\in\mathrm{aff}\left\{ \vec{0},y\right\} $, and hence $x\in\mathrm{aff}\,\overline{\Omega}$
(because $\mathrm{aff}\left\{ \overrightarrow{0},y\right\} \subset\mathrm{aff}\,\overline{\Omega}$).
Therefore $\mathbb{R}^{d}\subset\mathrm{aff}\,\overline{\Omega}$.
It follows that $\mathrm{aff}\,\overline{\Omega}=\mathbb{R}^{d}$
and $\dim\,\mathrm{aff}\,\overline{\Omega}=\dim\,\mathbb{R}^{d}=d$.
\begin{enumerate}
\item[{\crtcrossreflabel{$\mathbf{A2}$}[enu:A2].}] There exists a constant $R>0$ such that $\overline{\Omega}\subset B\left(\overrightarrow{0},R\right)$,
that is, $\overline{\Omega}$ is bounded.
\end{enumerate}
Moreover $\overline{\Omega}$ is not empty. Therefore $\overline{\Omega}$
is a polytope and
\[
\overline{\Omega}=\mathrm{conv}\left\{ v\left|\,v\in\mathcal{V}\right.\right\} ,
\]
where $\mathcal{V}$ is the finite set of vertices (extreme points)
of $\overline{\Omega}$. See Cor. 8.7 in \cite{ABrIP1983}.
\begin{enumerate}
\item[{\crtcrossreflabel{$\mathbf{A3}$}[enu:A3].}] $\partial\omega_{j}\cap\mathrm{int}\bigcap_{k\in\mathcal{K}\backslash\left\{ j\right\} }\overline{\omega_{k}}\neq\emptyset$
$\forall\;j\in\mathcal{K}$. Since $\overline{\Omega}$ is bounded,
$\left|\mathcal{K}\right|>1$. Moreover, $\dim\overline{\Omega}=3$.
Thus, this assumption is equivalent to impose that the representation
$\overline{\Omega}=\bigcap_{k\in\mathcal{K}}\overline{\omega_{k}}$
is irreducible. Then, the boundary of $\overline{\Omega}$ is $\bigcup_{k\in\mathcal{K}}\partial\omega_{k}\cap\overline{\Omega}$
and the faces of $\overline{\Omega}$ are the sets $\partial\omega_{k}\cap\overline{\Omega}$,
$k\in\mathcal{K}$. See Thm. 8.1, 8.2 in \cite{ABrIP1983}.
\item[{\crtcrossreflabel{$\mathbf{A4}$}[enu:A4].}] Each $v$ is contained in precisely $3$ faces. As $\dim\overline{\Omega}=3$,
this assumption is equivalent to impose that $\overline{\Omega}$
is simple. See Thm. 2.11 in \cite{ABrIP1983}. Given $v\in\mathcal{V}$,
denote by $\mathcal{I}_{v}\subset\mathcal{K}$ the subset of indices
such that $v\in\partial\omega_{k}$ $\forall$ $k\in\mathcal{I}_{v}$.
Clearly $\left|\mathcal{I}_{v}\right|=3$.
\end{enumerate}
%
(non-degeneracy of vertices) Let $v\in\mathcal{V}$ and let $\partial\omega_{i}\cap\overline{\Omega}$,
$\partial\omega_{j}\cap\overline{\Omega}$ and $\partial\omega_{k}\cap\overline{\Omega}$
be the faces containing to $v$. Suppose that $n_{k}=rn_{i}+sn_{j}$
with $r,s\geq0$. Then
\[
a_{k}=n_{k}v=rn_{i}v+sn_{j}v=ra_{i}+sa_{j}
\]
and
\[
n_{k}x=rn_{i}x+sn_{j}x\leq ra_{i}+sa_{j}=a_{k}\quad\forall\;x\in\overline{\omega_{i}}\cap\overline{\omega_{j}}.
\]
That is, $\overline{\omega_{i}}\cap\overline{\omega_{j}}\subset\overline{\omega_{k}}$.
Then $\overline{\omega_{i}}\cap\overline{\omega_{j}}\cap\overline{\omega_{k}}=\overline{\omega_{i}}\cap\overline{\omega_{j}}$.
Hence $Q$ is reductible because $Q=\bigcap_{k'\in\mathcal{K}\backslash\left\{ k\right\} }\overline{\omega_{k'}}$
($\rightarrow\leftarrow$). If $r,s\leq0$, then $a_{k}\leq0$ ($\rightarrow\leftarrow$).
In the case $r>0$ and $s<0$, it suffices to apply the same arguments
to $n_{i}=\left(1/r\right)n_{k}+\left(-s/r\right)n_{j}$. Therefore
the vectors $n_{i},n_{j},n_{k}$ are linearly independent. We write
this as $\mathrm{rank}\left(n_{i},n_{j},n_{k}\right)^{\top}=3$. We
can write
\[
\partial\omega_{i}\cap\partial\omega_{j}\cap\partial\omega_{k}=\left\{ v\right\} .
\]

\begin{prop}
$\exists$ $\tau>0$ such that $\forall$ $v\in\mathcal{V}$ $\exists!$
$z_{v}:\left(-\tau,\tau\right)\rightarrow\mathbb{R}^{d}$ satisfying
$z_{v}\left(0\right)=v$ and
\[
\partial\omega_{i}\left(t\right)\cap\partial\omega_{j}\left(t\right)\cap\partial\omega_{k}\left(t\right)=\left\{ z_{v}\left(t\right)\right\} \quad\forall\;t\in\left(-\tau,\tau\right),
\]
where $\mathcal{I}_{v}=\left\{ i,j,k\right\} $. Moreover, each $z_{v}$
is $C^{\infty}$ on $\left(-\tau,\tau\right)$ and
\[
z_{v}^{\prime}\left(0\right)=\left(n_{i},n_{j},n_{k}\right)^{-\top}\left(\left(a_{i}^{\prime}\left(0\right),a_{j}^{\prime}\left(0\right),a_{k}^{\prime}\left(0\right)\right)^{\top}-\left(n_{i}^{\prime}\left(0\right),n_{j}^{\prime}\left(0\right),n_{k}^{\prime}\left(0\right)\right)^{\top}v\right).
\]
\end{prop}
%
\begin{proof}
. $F:\mathbb{R}^{3}\times\mathbb{R}\rightarrow\mathbb{R}$
\[
F\left(x,t\right)=\left(\begin{array}{c}
\widehat{\phi}_{i}\left(x,t\right)\\
\widehat{\phi}_{j}\left(x,t\right)\\
\widehat{\phi}_{k}\left(x,t\right)
\end{array}\right)
\]
\[
F\left(v,0\right)=0
\]
\[
D_{x}F\left(x,t\right)=\left(\begin{array}{c}
\nabla_{x}\widehat{\phi}_{i}\left(x,t\right)^{\top}\\
\nabla_{x}\widehat{\phi}_{j}\left(x,t\right)^{\top}\\
\nabla_{x}\widehat{\phi}_{k}\left(x,t\right)^{\top}
\end{array}\right)=\left(\begin{array}{c}
n_{i}\left(t\right)^{\top}\\
n_{j}\left(t\right)^{\top}\\
n_{k}\left(t\right)^{\top}
\end{array}\right)
\]
\[
D_{x}F\left(v,0\right)=\left(\begin{array}{c}
n_{i}{}^{\top}\\
n_{j}{}^{\top}\\
n_{k}{}^{\top}
\end{array}\right)=\left(n_{k}\right)_{k\in\mathcal{I}_{v}}^{\top}
\]
\[
\partial_{t}F\left(x,t\right)=\left(\begin{array}{c}
\partial_{t}\widehat{\phi}_{i}\left(x,t\right)\\
\partial_{t}\widehat{\phi}_{j}\left(x,t\right)\\
\partial_{t}\widehat{\phi}_{k}\left(x,t\right)
\end{array}\right)=\left(\begin{array}{c}
x\cdot n_{i}^{\prime}\left(t\right)-a_{i}^{\prime}\left(t\right)\\
x\cdot n_{j}^{\prime}\left(t\right)-a_{j}^{\prime}\left(t\right)\\
x\cdot n_{k}^{\prime}\left(t\right)-a_{k}^{\prime}\left(t\right)
\end{array}\right)
\]
\[
\partial_{t}F\left(v,0\right)=\left(\begin{array}{c}
v\cdot n_{i}^{\prime}\left(0\right)-a_{i}^{\prime}\left(0\right)\\
v\cdot n_{j}^{\prime}\left(0\right)-a_{j}^{\prime}\left(0\right)\\
v\cdot n_{k}^{\prime}\left(0\right)-a_{k}^{\prime}\left(0\right)
\end{array}\right)=\left(\begin{array}{c}
n_{i}^{\prime}\left(0\right)^{\top}\\
n_{j}^{\prime}\left(0\right)^{\top}\\
n_{k}^{\prime}\left(0\right)^{\top}
\end{array}\right)v-\left(\begin{array}{c}
a_{i}^{\prime}\left(0\right)\\
a_{j}^{\prime}\left(0\right)\\
a_{k}^{\prime}\left(0\right)
\end{array}\right)
\]
\end{proof}
%
(non-degeneracy of edges) If $n_{i},n_{j}$ are the normal vectors
of the faces of $\overline{\Omega}$ containing an edge, then $\mathrm{rank}\left(n_{i},n_{j}\right)^{\top}=2$
(non-degeneracy of edges). Moreover, given $k\in\mathcal{K}$, $v\cdot n_{k}<a_{k}$
for all $v\in\mathcal{V}\backslash\mathcal{W}$, where $\mathcal{W}$
is the subset of vertices contained in $\left\{ x\in\mathbb{R}^{3}\left|\,x\cdot n_{k}-a_{k}=0\right.\right\} $.
\begin{prop}
$\overline{\Omega_{t}}$ is simple convex polyhedral for small $t$.
\end{prop}
%
\begin{proof}
Since $a_{k}\left(t\right)$ is smooth, there exists $\tau_{0}>0$
such that $a_{k}\left(t\right)>0$ for all $t\in\left(-\tau_{0},\tau_{0}\right)$.
Let $n_{i},n_{j},n_{k}$ be the normal vectors of the faces of $\overline{\Omega}$
containing a vertex $v$. As $\mathrm{rank}\left(n_{i},n_{j},n_{k}\right)^{\top}=3$,
we can suppose that
\[
\det\left(n_{i}\left(0\right),n_{j}\left(0\right),n_{k}\left(0\right)\right)^{\top}>0.
\]
Since the determinant is a continuous functions, there exists $\tau_{v}>0$
such that
\[
\det\left(n_{i}\left(t\right),n_{j}\left(t\right),n_{k}\left(t\right)\right)^{\top}>0\quad\forall\;t\in\left(-\tau_{v},\tau_{v}\right),
\]
and then $\mathrm{rank}\left(n_{i}\left(t\right),n_{j}\left(t\right),n_{k}\left(t\right)\right)^{\top}=3$
for all $t\in\left(-\tau_{v},\tau_{v}\right)$. Let $n_{i},n_{j}$
be the normal vectors of the faces of $\overline{\Omega}$ containing
an edge $e$. As $\mathrm{rank}\left(n_{i},n_{j}\right)^{\top}=2$
if and only if $-1<n_{i}\cdot n_{j}<1$, we have
\[
1-\left|n_{i}\left(0\right)\cdot n_{j}\left(0\right)\right|>0.
\]
Since the function $\lambda\mapsto1-\left|\lambda\right|$ is continuous,
there exists $\tau_{e}>0$ such that
\[
1-\left|n_{i}\left(t\right)\cdot n_{j}\left(t\right)\right|>0\quad\forall\;t\in\left(-\tau_{e},\tau_{e}\right),
\]
and then $\mathrm{rank}\left(n_{i}\left(t\right),n_{j}\left(t\right)\right)^{\top}=2$
for all $t\in\left(-\tau_{e},\tau_{e}\right)$. Let $\tau\leq\tau_{0},\min_{v}\tau_{v},\min_{e}\tau_{e}$.
Thus, the domain $\overline{\Omega_{t}}$ is the intersection of halfspaces
$\overline{\omega_{k}\left(t\right)}$ containing to the origen which
form non-degenerate vertices and edges for all $t\in\left(-\tau,\tau\right)$.
Therefore $\overline{\Omega_{t}}$ is a simple convex polyhedral for
all $t\in\left(-\tau,\tau\right)$.
\end{proof}
\begin{rem}
The polyhedral $\Omega_{t}$ in Theorem has the same number of vertices,
edges and faces that $\Omega$.
\end{rem}
In particular,
\[
\int_{\partial\Omega}\frac{\left|\partial_{n}u\right|^{2}}{2}\theta_{n}=\sum_{k\in\mathcal{K}}\dot{a}_{k}\left(0\right)\int_{\partial\omega_{k}\cap\overline{\Omega}}\frac{\left|\partial_{n}u\right|^{2}}{2}-\int_{\partial\omega_{k}\cap\overline{\Omega}}\frac{\left|\partial_{n}u\right|^{2}}{2}x\cdot\dot{n}_{k}\left(0\right)
\]
Case 2D:
\[
n_{k}\left(t\right)=\left(\begin{array}{c}
\cos\left(\alpha_{k}+t\delta\alpha_{k}\right)\\
\sin\left(\alpha_{k}+t\delta\alpha_{k}\right)
\end{array}\right),
\]
\[
a_{k}\left(t\right)=\left(\lambda_{k}+t\delta\lambda_{k}\right)n_{k}\left(t\right)\cdot n_{k}\left(0\right).
\]
\[
\dot{n}_{k}\left(0\right)=\delta\alpha_{k}\left(\begin{array}{c}
-\sin\left(\alpha_{k}\right)\\
\cos\left(\alpha_{k}\right)
\end{array}\right)
\]
\[
\dot{a}_{k}\left(0\right)=\delta\lambda_{k}
\]
\[
\int_{\partial\Omega}\frac{\left|\partial_{n}u\right|^{2}}{2}\theta_{n}=\sum_{k\in\mathcal{K}}\delta\lambda_{k}\int_{f_{k}}\frac{\left|\partial_{n}u\right|^{2}}{2}-\delta\alpha_{k}\int_{f_{k}}\frac{\left|\partial_{n}u\right|^{2}}{2}x\cdot\left(\begin{array}{c}
-\sin\left(\alpha_{k}\right)\\
\cos\left(\alpha_{k}\right)
\end{array}\right)
\]
Caso 3D:
\[
n_{k}\left(t\right)=\left(\begin{array}{c}
\cos\left(\alpha_{k}+t\delta\alpha_{k}\right)\sin\left(\beta_{k}+t\delta\beta_{k}\right)\\
\sin\left(\alpha_{k}+t\delta\alpha_{k}\right)\sin\left(\beta_{k}+t\delta\beta_{k}\right)\\
\cos\left(\beta_{k}+t\delta\beta_{k}\right)
\end{array}\right),
\]
\[
a_{k}\left(t\right)=\left(\lambda_{k}+t\delta\lambda_{k}\right)n_{k}\left(t\right)\cdot n_{k}\left(0\right).
\]
\[
\dot{n}_{k}\left(0\right)=\delta\alpha_{k}\left(\begin{array}{c}
-\sin\left(\alpha_{k}\right)\\
\cos\left(\alpha_{k}\right)
\end{array}\right)
\]
\[
\dot{n}_{k}\left(0\right)=\delta\alpha_{k}\left(\begin{array}{c}
-\sin\left(\alpha_{k}\right)\sin\left(\beta_{k}\right)\\
\cos\left(\alpha_{k}\right)\sin\left(\beta_{k}\right)\\
0
\end{array}\right)+\delta\beta_{k}\left(\begin{array}{c}
\cos\left(\alpha_{k}\right)\cos\left(\beta_{k}\right)\\
\sin\left(\alpha_{k}\right)\cos\left(\beta_{k}\right)\\
-\sin\left(\beta_{k}\right)
\end{array}\right)
\]
\[
\dot{a}_{k}\left(0\right)=\delta\lambda_{k}
\]
\[
\int_{\partial\Omega}\frac{\left|\partial_{n}u\right|^{2}}{2}\theta_{n}=\sum_{k\in\mathcal{K}}\delta\lambda_{k}\int_{f_{k}}\frac{\left|\partial_{n}u\right|^{2}}{2}-\delta\alpha_{k}\int_{f_{k}}\frac{\left|\partial_{n}u\right|^{2}}{2}x\cdot\mathbf{i}_{k}-\delta\beta_{k}\int_{f_{k}}\frac{\left|\partial_{n}u\right|^{2}}{2}x\cdot\mathbf{j}_{k}
\]
\[
\delta\lambda_{k}=-\int_{f_{k}}\frac{\left|\partial_{n}u\right|^{2}}{2}
\]
\[
\delta\alpha_{k}=\int_{f_{k}}\frac{\left|\partial_{n}u\right|^{2}}{2}x\cdot\mathbf{i}_{k}
\]
\[
\delta\beta_{k}=\int_{f_{k}}\frac{\left|\partial_{n}u\right|^{2}}{2}x\cdot\mathbf{j}_{k}
\]
\[
\int_{\partial\Omega}\frac{\left|\partial_{n}u\right|^{2}}{2}\theta_{n}<0
\]
\[
\lambda_{k}^{1}=\lambda_{k}^{0}-\tilde{t}\int_{f_{k}}\frac{\left|\partial_{n}u\right|^{2}}{2}
\]
\[
\alpha_{k}^{1}=\alpha_{k}^{0}+\tilde{t}\int_{f_{k}}\frac{\left|\partial_{n}u\right|^{2}}{2}x\cdot\mathbf{i}_{k}
\]
\[
\beta_{k}^{1}=\beta_{k}^{0}+\tilde{t}\int_{f_{k}}\frac{\left|\partial_{n}u\right|^{2}}{2}x\cdot\mathbf{j}_{k}
\]
Assumptions.
\begin{itemize}
\item Given $k\in\mathcal{K}$, $\overline{\Omega}\backslash\partial\omega_{k}\subset\omega_{k}$. 
\end{itemize}
%
Let $v\in\mathcal{V}$ and $n_{i},n_{j},n_{k}$ be the normal vectors
of the faces of $\overline{\Omega}$ containing $v$. Then $\widehat{\phi}_{i}\left(v,0\right)=\widehat{\phi}_{j}\left(v,0\right)=\widehat{\phi}_{k}\left(v,0\right)=0$.
Since $\mathrm{rank}\left(n_{i},n_{j},n_{k}\right)^{\top}=3$, there
exist a $\tau_{v}$ and a unique function $z_{v}:\left(-\tau_{v},\tau_{v}\right)\rightarrow\mathbb{R}^{3}$
such that $z_{v}\left(0\right)=v$ and 
\[
\widehat{\phi}_{i}\left(z_{v}\left(t\right),t\right)=\widehat{\phi}_{j}\left(z_{v}\left(t\right),t\right)=\widehat{\phi}_{k}\left(z_{v}\left(t\right),t\right)=0\quad\forall\;t\in\left(-\tau_{v},\tau_{v}\right).
\]

\begin{align*}
e_{i,j}\left(t\right) & \coloneqq\left\{ x\in\mathbb{R}^{3}\left|\,\widehat{\phi}_{i}\left(x,t\right)=\widehat{\phi}_{j}\left(x,t\right)=0,\;\widehat{\phi}_{k}\left(x,t\right)<0\;\forall\,k\in\mathcal{K}\backslash\left\{ i,j\right\} \right.\right\} ,\\
 & =\partial\omega_{i}\left(t\right)\cap\partial\omega_{j}\left(t\right)\cap\left(\cap_{k\in\mathcal{K}\backslash\left\{ i,j\right\} }\omega_{k}\left(t\right)\right)
\end{align*}
\begin{align*}
f_{k}\left(t\right) & \coloneqq\left\{ x\in\mathbb{R}^{3}\left|\,\widehat{\phi}_{k}\left(x,t\right)=0,\;\widehat{\phi}_{j}\left(x,t\right)<0\;\forall\,j\in\mathcal{K}\backslash\left\{ k\right\} \right.\right\} .\\
 & =\partial\omega_{k}\left(t\right)\cap\left(\cap_{j\in\mathcal{K}\backslash\left\{ k\right\} }\omega_{j}\left(t\right)\right)
\end{align*}
Denote $\omega_{k}\coloneqq\omega_{k}\left(0\right)$, $e_{k,k^{\prime}}\coloneqq e_{k,k^{\prime}}\left(0\right)$,
$f_{k}\coloneqq f_{k}\left(0\right)$ and $\Omega\coloneqq\Omega_{0}$.
\begin{defn}
D
\end{defn}
\begin{thm}
There exist $\tau>0$ and a mapping $T:\overline{\Omega}\times\left(-\tau,\tau\right)\rightarrow\mathbb{R}^{d}$
satisfying $T\left(\Omega,t\right)=\Omega_{t}$, $T\left(\partial\Omega,t\right)=\partial\Omega_{t}$
and such that $T\left(\cdot,t\right):\overline{\Omega}\rightarrow\overline{\Omega_{t}}$
is bi-Lipschitz for all $t\in\left(-\tau,\tau\right)$. Furthermore,
we have
\begin{equation}
\theta\left(x\right)\cdot\nu\left(x\right)=\dot{a}_{k}\left(0\right)-x\cdot\dot{n}_{k}\left(0\right)\quad\forall\;x\in\mathrm{int}\,\partial\omega_{k}\cap\overline{\Omega},
\end{equation}
where $\theta\coloneqq\partial_{t}T\left(\cdot,0\right)$ and $\nu$
is the outward unit normal vector to $\Omega$.
\end{thm}
%
\begin{proof}
Let $T:\overline{\Omega}\times\left(-\tau,\tau\right)\rightarrow\mathbb{R}^{3}$
be the vector-valued function defined by 
\[
T\left(x,t\right)\coloneqq\sum_{v\in\mathcal{V}}b_{v}\left(x\right)z_{v}\left(t\right).
\]
Let $v\in\mathcal{V}$. We have
\[
T\left(v,t\right)=\sum_{w\in\mathcal{V}}b_{w}\left(v\right)z_{w}\left(t\right)=\sum_{w\in\mathcal{V}}\delta_{wv}z_{w}\left(t\right)=z_{v}\left(t\right).
\]
Let $e_{i,j}$ be an edge and $v,w$ be its vertices. Let $x\in e_{i,j}$.
We have
\[
T\left(x,t\right)=\sum_{u\in\mathcal{V}}b_{u}\left(x\right)z_{u}\left(t\right)=b_{v}\left(x\right)z_{v}\left(t\right)+b_{w}\left(x\right)z_{w}\left(t\right)
\]
with $b_{v}\left(x\right),b_{w}\left(x\right)>0$. Thus, given $k\in\mathcal{K}$
it follows that
\[
\hat{\phi}_{k}\left(T\left(x,t\right),t\right)=b_{v}\left(x\right)z_{v}\left(t\right)\cdot n_{k}\left(t\right)+b_{w}\left(x\right)z_{w}\left(t\right)\cdot n_{k}\left(t\right)-a_{k}\left(t\right).
\]
If $k\in\left\{ i,j\right\} $, then
\[
\hat{\phi}_{k}\left(T\left(x,t\right),t\right)=\left(b_{v}\left(x\right)+b_{w}\left(x\right)\right)a_{k}\left(t\right)-a_{k}\left(t\right)=0
\]
because $z_{v}\left(t\right),z_{w}\left(t\right)\in\partial\omega_{k}\left(t\right)$.
Hence $T\left(e_{i,j},t\right)\subset\partial\omega_{i}\left(t\right)\cap\partial\omega_{j}\left(t\right)$.
If $k\in\mathcal{K}\backslash\left\{ i,j\right\} $ we have three
cases: $\partial\omega_{k}\left(t\right)$ is the third plane containing
to $z_{v}\left(t\right)$, $\partial\omega_{k}\left(t\right)$ is
the third plane containing to $z_{w}\left(t\right)$ and $\partial\omega_{k}\left(t\right)$
is a plane that does not contain neither $z_{v}\left(t\right)$ and
$z_{v}\left(t\right)$. In the first case
\begin{align*}
\hat{\phi}_{k}\left(T\left(x,t\right),t\right) & =b_{v}\left(x\right)a_{k}\left(t\right)+b_{w}\left(x\right)z_{w}\left(t\right)\cdot n_{k}\left(t\right)-a_{k}\left(t\right)\\
 & <b_{v}\left(x\right)a_{k}\left(t\right)+b_{w}\left(x\right)a_{k}\left(t\right)-a_{k}\left(t\right)\\
 & =\left(b_{v}\left(x\right)+b_{w}\left(x\right)\right)a_{k}\left(t\right)-a_{k}\left(t\right)=0.
\end{align*}
In the second case
\begin{align*}
\hat{\phi}_{k}\left(T\left(x,t\right),t\right) & =b_{v}\left(x\right)z_{v}\left(t\right)\cdot n_{k}\left(t\right)+b_{w}\left(x\right)a_{k}\left(t\right)-a_{k}\left(t\right)\\
 & <b_{v}\left(x\right)a_{k}\left(t\right)+b_{w}\left(x\right)a_{k}\left(t\right)-a_{k}\left(t\right)\\
 & =\left(b_{v}\left(x\right)+b_{w}\left(x\right)\right)a_{k}\left(t\right)-a_{k}\left(t\right)=0.
\end{align*}
Finally, in the third case,
\begin{align*}
\hat{\phi}_{k}\left(T\left(x,t\right),t\right) & =b_{v}\left(x\right)z_{v}\left(t\right)\cdot n_{k}\left(t\right)+b_{w}\left(x\right)z_{w}\left(t\right)\cdot n_{k}\left(t\right)-a_{k}\left(t\right)\\
 & <b_{v}\left(x\right)a_{k}\left(t\right)+b_{w}\left(x\right)a_{k}\left(t\right)-a_{k}\left(t\right)\\
 & =\left(b_{v}\left(x\right)+b_{w}\left(x\right)\right)a_{k}\left(t\right)-a_{k}\left(t\right)=0.
\end{align*}
Hence $T\left(e_{i,j},t\right)\subset\left(\cap_{k\in\mathcal{K}\backslash\left\{ i,j\right\} }\omega_{k}\left(t\right)\right)$.
We conclude that $T\left(e_{i,j},t\right)\subset e_{i,j}\left(t\right)$.
Let $y\in e_{i,j}\left(t\right)$. Then $y=\lambda z_{v}\left(t\right)+\left(1-\lambda\right)z_{w}\left(t\right)$
for some $0<\lambda<1$. Let $x=w+\lambda\left(v-w\right)$. We have
\begin{align*}
T\left(x,t\right) & =b_{v}\left(x\right)z_{v}\left(t\right)+b_{w}\left(x\right)z_{w}\left(t\right)\\
 & =\frac{\left\langle x-w,v-w\right\rangle }{\left|\overline{e_{i,j}}\right|^{2}}z_{v}\left(t\right)+\frac{\left\langle x-v,w-v\right\rangle }{\left|\overline{e_{i,j}}\right|^{2}}z_{w}\left(t\right)\\
 & =\frac{\left\langle \lambda\left(v-w\right),v-w\right\rangle }{\left|\overline{e_{i,j}}\right|^{2}}z_{v}\left(t\right)+\frac{\left\langle \left(1-\lambda\right)\left(w-v\right),w-v\right\rangle }{\left|\overline{e_{i,j}}\right|^{2}}z_{w}\left(t\right)\\
 & =\lambda\frac{\left\langle v-w,v-w\right\rangle }{\left|\overline{e_{i,j}}\right|^{2}}z_{v}\left(t\right)+\left(1-\lambda\right)\frac{\left\langle w-v,w-v\right\rangle }{\left|\overline{e_{i,j}}\right|^{2}}z_{w}\left(t\right)\\
 & =y.
\end{align*}
Therefore, $T\left(e_{i,j},t\right)=e_{i,j}\left(t\right)$. Moreover,
if $T\left(x_{1},t\right)=T\left(x_{2},t\right)$ for $x_{1},x_{2}\in e_{i,j}$,
\begin{align*}
b_{v}\left(x_{1}\right)z_{v}\left(t\right)+\left(1-b_{v}\left(x_{1}\right)\right)z_{w}\left(t\right) & =b_{v}\left(x_{2}\right)z_{v}\left(t\right)+\left(1-b_{v}\left(x_{2}\right)\right)z_{w}\left(t\right)\\
b_{v}\left(x_{1}\right)\left(z_{v}\left(t\right)-z_{w}\left(t\right)\right)+z_{w}\left(t\right) & =b_{v}\left(x_{2}\right)\left(z_{v}\left(t\right)-z_{w}\left(t\right)\right)+z_{w}\left(t\right)\\
b_{v}\left(x_{1}\right)\left(z_{v}\left(t\right)-z_{w}\left(t\right)\right) & =b_{v}\left(x_{2}\right)\left(z_{v}\left(t\right)-z_{w}\left(t\right)\right)
\end{align*}
It follows that $b_{v}\left(x_{1}\right)=b_{v}\left(x_{2}\right)$,
that is
\begin{align*}
\frac{\left\langle x_{1}-w,v-w\right\rangle }{\left|\overline{e_{i,j}}\right|^{2}} & =\frac{\left\langle x_{2}-w,v-w\right\rangle }{\left|\overline{e_{i,j}}\right|^{2}}\\
\left\langle x_{1}-w,v-w\right\rangle  & =\left\langle x_{2}-w,v-w\right\rangle 
\end{align*}
and $\left\langle x_{1}-x_{2},v-w\right\rangle =0$. Since $x_{1}-x_{2}$
is parallel to $v-w$, $x_{1}=x_{2}$. Therefore $T$ is bijective
on $e_{i,j}$. Let $f_{k}$ be a face and $\mathcal{W}$ be the set
of its vertices. Let $x\in f_{k}$. We have
\[
T\left(x,t\right)=\sum_{w\in\mathcal{V}}b_{w}\left(x\right)z_{w}\left(t\right)=\sum_{w\in\mathcal{W}}b_{w}\left(x\right)z_{w}\left(t\right)
\]
and
\begin{align*}
\hat{\phi}_{k}\left(T\left(x,t\right),t\right) & =\sum_{w\in\mathcal{W}}b_{w}\left(x\right)z_{w}\left(t\right)\cdot n_{k}\left(t\right)-a_{k}\left(t\right)\\
 & =\left(\sum_{w\in\mathcal{W}}b_{w}\left(x\right)\right)a_{k}\left(t\right)-a_{k}\left(t\right)=0.
\end{align*}
Hence $T\left(f_{k},t\right)\subset\partial\omega_{k}\left(t\right)$.
Let $j\in\mathcal{K}\backslash\left\{ k\right\} $. There are two
cases: $\partial\omega_{j}\left(t\right)$ is a plane touching $f_{k}$
on the edge with vertices $u,v$ and otherwise. In the first case,
there are at least one vertex $w\in\mathcal{W}$ such that $w\in\omega_{j}\left(t\right)$
(it is because a face has least one vertex out the plane $\partial\omega_{j}\left(t\right)$).
Then
\begin{align*}
\hat{\phi}_{j}\left(T\left(x,t\right),t\right) & =\sum_{w\in\mathcal{W}}b_{w}\left(x\right)z_{w}\left(t\right)\cdot n_{j}\left(t\right)-a_{j}\left(t\right)\\
 & =\sum_{w\in\mathcal{W}\backslash\left\{ u,v\right\} }b_{w}\left(x\right)z_{w}\left(t\right)\cdot n_{j}\left(t\right)+\left(b_{u}\left(x\right)+b_{v}\left(x\right)\right)a_{j}\left(t\right)-a_{j}\left(t\right)\\
 & =\sum_{w\in\mathcal{W}\backslash\left\{ u,v\right\} }b_{w}\left(x\right)\left(z_{w}\left(t\right)\cdot n_{j}\left(t\right)-a_{j}\left(t\right)\right)+a_{j}\left(t\right)-a_{j}\left(t\right)\\
 & =\sum_{w\in\mathcal{W}\backslash\left\{ u,v\right\} }b_{w}\left(x\right)\left(z_{w}\left(t\right)\cdot n_{j}\left(t\right)-a_{j}\left(t\right)\right)
\end{align*}
since $u\cdot n_{j}\left(t\right)=v\cdot n_{j}\left(t\right)=a_{j}\left(t\right)$.
We know that $z_{w}\left(t\right)\cdot n_{j}\left(t\right)-a_{j}\left(t\right)<0$
for all $w\in\mathcal{W}\backslash\left\{ u,v\right\} $. Is there
some $w\in\mathcal{W}$ such that $b_{w}\left(x\right)>0$? Yes! Indeed,
since $x$ is not in the edge of vertices $u,v$, there exists some
$w\in\mathcal{V}\backslash\left\{ u,v\right\} $ such that $b_{w}\left(x\right)>0$.
But, since $x$ is in the face $f_{k}$, $b_{w}\left(x\right)=0$
for all $w\in\mathcal{V}\backslash\mathcal{W}$. Then there is some
$w\in\mathcal{W}\backslash\left\{ u,v\right\} $ such that $b_{w}\left(x\right)>0$.
Thus
\[
\hat{\phi}_{j}\left(T\left(x,t\right),t\right)=\sum_{w\in\mathcal{W}\backslash\left\{ u,v\right\} }b_{w}\left(x\right)\left(z_{w}\left(t\right)\cdot n_{j}\left(t\right)-a_{j}\left(t\right)\right)<0.
\]
In the second case, 
\begin{align*}
\hat{\phi}_{j}\left(T\left(x,t\right),t\right) & =\sum_{w\in\mathcal{W}}b_{w}\left(x\right)z_{w}\left(t\right)\cdot n_{j}\left(t\right)-a_{j}\left(t\right)\\
 & =\sum_{w\in\mathcal{W}}b_{w}\left(x\right)\left(z_{w}\left(t\right)\cdot n_{j}\left(t\right)-a_{j}\left(t\right)\right)
\end{align*}
We know that $z_{w}\left(t\right)\cdot n_{j}\left(t\right)-a_{j}\left(t\right)<0$.
If all $b_{w}\left(x\right)=0$ for all $w\in\mathcal{W}$ then $x=0\in\partial\omega_{k}\left(t\right)$.
Comtradiction!
\[
\hat{\phi}_{j}\left(T\left(x,t\right),t\right)<0
\]
We conclude that $T\left(x,t\right)\in\partial\omega_{k}\left(t\right)\cap\left(\cap_{j\in\mathcal{K}\backslash\left\{ k\right\} }\omega_{j}\left(t\right)\right)$,
that is, $T\left(f_{k},t\right)\subset f_{k}\left(t\right)$. Therefore
\[
T\left(\overline{f_{k}},t\right)\subset\overline{f_{k}\left(t\right)},T\left(\partial f_{k\partial},t\right)=\partial f_{k}\left(t\right)
\]
and $T$ is injective on $\partial f_{k\partial}$.�

~

~

Let $x\in\cap_{k\in\mathcal{K}}\omega_{k}$. Let $k\in\mathcal{K}$
and $\mathcal{W}$ be the set of vertices of the face $f_{k}$. We
have
\begin{align*}
\hat{\phi}_{k}\left(T\left(x,t\right),t\right) & =\sum_{v\in\mathcal{V}}b_{v}\left(x\right)z_{v}\left(t\right)\cdot n_{k}\left(t\right)-a_{k}\left(t\right)\\
 & =\left(\sum_{w\in\mathcal{W}}b_{w}\left(x\right)-1\right)a_{k}\left(t\right)+\sum_{v\in\mathcal{V}\backslash\mathcal{W}}b_{v}\left(x\right)z_{v}\left(t\right)\cdot n_{k}\left(t\right)\\
 & =-\sum_{v\in\mathcal{V}\backslash\mathcal{W}}b_{v}\left(x\right)a_{k}\left(t\right)+\sum_{v\in\mathcal{V}\backslash\mathcal{W}}b_{v}\left(x\right)z_{v}\left(t\right)\cdot n_{k}\left(t\right)\\
 & =\sum_{v\in\mathcal{V}\backslash\mathcal{W}}b_{v}\left(x\right)\left(z_{v}\left(t\right)\cdot n_{k}\left(t\right)-a_{k}\left(t\right)\right)\\
 & =\sum_{v\in\mathcal{V}\backslash\mathcal{W}}b_{v}\left(x\right)\hat{\phi}_{k}\left(z_{v}\left(t\right),t\right)
\end{align*}
We know that there exists $w\in\mathcal{V}\backslash\mathcal{W}$
such that $b_{w}\left(x\right)>0$. 

Since $b_{v}\left(x\right)>0$ for all $v$ and $\hat{\phi}_{k}\left(z_{v}\left(t\right),t\right)<0$
for all $v\in\mathcal{V}\backslash\mathcal{W}$ (it must be hip.),
$\hat{\phi}_{k}\left(T\left(x,t\right),t\right)<0$. Hence $T\left(x,t\right)\in\cap_{k\in\mathcal{K}}\omega_{k}\left(t\right)$.
\end{proof}

\section{Injectivity}

\[
T\left(x,t\right)=x+\sum_{v\in\mathcal{V}}b_{v}\left(x\right)\left(z_{v}\left(t\right)-v\right)
\]
\[
T\left(x,t\right)-T\left(y,t\right)=x-y+\sum_{v\in\mathcal{V}}\left(b_{v}\left(x\right)-b_{v}\left(y\right)\right)\left(z_{v}\left(t\right)-v\right)
\]
It holds that
\[
\left\Vert x-y\right\Vert -\left\Vert \sum_{v\in\mathcal{V}}\left(b_{v}\left(x\right)-b_{v}\left(y\right)\right)\left(z_{v}\left(t\right)-v\right)\right\Vert \leq\left\Vert T\left(x,t\right)-T\left(y,t\right)\right\Vert \leq\left\Vert x-y\right\Vert +\left\Vert \sum_{v\in\mathcal{V}}\left(b_{v}\left(x\right)-b_{v}\left(y\right)\right)\left(z_{v}\left(t\right)-v\right)\right\Vert 
\]
if
\[
\left|b_{v}\left(x\right)-b_{v}\left(y\right)\right|\leq\left\Vert \nabla b_{v}\left(z\left(x,y,v\right)\right)\right\Vert \left\Vert x-y\right\Vert 
\]
Then
\[
\left\Vert x-y\right\Vert \left(1-\sum_{v\in\mathcal{V}}\left\Vert \nabla b_{v}\left(z\left(x,y,v\right)\right)\right\Vert \left\Vert z_{v}\left(t\right)-v\right\Vert \right)\leq\left\Vert T\left(x,t\right)-T\left(y,t\right)\right\Vert \leq\left\Vert x-y\right\Vert \left(1+\sum_{v\in\mathcal{V}}\left\Vert \nabla b_{v}\left(z\left(x,y,v\right)\right)\right\Vert \left\Vert z_{v}\left(t\right)-v\right\Vert \right)
\]
Assume that
\[
\left\Vert \nabla b_{v}\left(z\left(x,y,v\right)\right)\right\Vert \leq C
\]
for all $v$ and all $x,y\in\overline{\Omega}$. Then
\[
\left\Vert x-y\right\Vert \left(1-C\sum_{v\in\mathcal{V}}\left\Vert z_{v}\left(t\right)-v\right\Vert \right)\leq\left\Vert T\left(x,t\right)-T\left(y,t\right)\right\Vert \leq\left\Vert x-y\right\Vert \left(1+C\sum_{v\in\mathcal{V}}\left\Vert z_{v}\left(t\right)-v\right\Vert \right)
\]
and if 
\[
\sum_{v\in\mathcal{V}}\left\Vert z_{v}\left(t\right)-v\right\Vert <\frac{1}{C}
\]
From this we deduce that $T$ is bi

Rewriteen $T$ as
\[
T\left(x,t\right)=x+\sum_{v\in\mathcal{V}}b_{v}\left(x\right)\left(z_{v}\left(t\right)-v\right)
\]
it is easy to calculate its Jacobian matrix
\[
D_{x}T\left(x,t\right)=I+\sum_{v\in\mathcal{V}}\left(z_{v}\left(t\right)-v\right)\otimes\nabla b_{v}\left(x\right).
\]
Let $d:\overline{\Omega}\times\mapsto\mathbb{R}$ be the function
defined by
\[
d\left(x,t\right)\coloneqq\mathrm{det}\left(I+\sum_{v\in\mathcal{V}}\left(z_{v}\left(t\right)-v\right)\otimes\nabla b_{v}\left(x\right)\right).
\]
Fix an arbitrary $x\in\overline{\Omega}$. Since $d\left(x,\cdot\right)$
is a continuous function and 
\[
d\left(x,0\right)=\mathrm{det}\left(I+\sum_{v\in\mathcal{V}}\left(z_{v}\left(0\right)-v\right)\otimes\nabla b_{v}\left(x\right)\right)=\mathrm{det}\,I=1>0,
\]
it follows that $\mathrm{det}\,D_{x}T\left(x,t\right)=d\left(x,t\right)>0$
for small $t$. Therefore, one can conclude that $T\left(\cdot,t\right)$
is locally injective for small $t$.

This idea was taken from \cite{TScBM2013}\href{file:references/TScBM2013.pdf}{�}.
Recall that non zero determinant of $D_{x}T\left(x,t\right)$ implies
local injectivity. To obtain global injectivity consider \cite{MFlIP2010}\href{file:references/MFlIP2010.pdf}{�}
\cite{HKeMJ1971}\href{file:references/HKeMJ1971.pdf}{�}.

\section{Surjectivity}

h1: $U,V\subset\mathbb{R}^{n}$ are closed, bounded and convex

h2: $f:U\rightarrow V$ is continuous

h2: $f\left(\mathrm{int}U\right)\subset\mathrm{int}V$

h3: $f:\partial U\rightarrow\partial V$ is an homeomorphism

Then $f$ is surjective.

Proof.

Let $p\in\mathrm{int}U$.

Let $H:I\times\partial V\rightarrow V$ defined by

$H\left(t,y\right)=f\left(\left(1-t\right)f^{-1}\left(y\right)+tp\right)$,
which is well-defined, continuous, $H\left(0,y\right)=y$ and $H\left(1,y\right)=f\left(p\right)$.
The homotopy $H$ continuously contracts the boundary $\partial V$
to a $f\left(p\right)$. Then, for all $z\in V$, is there some $x$
such that $H\left(t,y\right)=z$?

\bibliographystyle{plain}
\bibliography{references}

\end{document}
