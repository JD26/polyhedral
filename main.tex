\documentclass[english]{article}
\usepackage[T1]{fontenc}
\usepackage[latin9]{inputenc}
\usepackage{geometry}
\geometry{verbose,lmargin=3cm,rmargin=3cm}
\setlength{\parindent}{0bp}
\usepackage{amsthm}
\usepackage{textcomp}
\usepackage{mathtools}
\usepackage{amsmath}
\usepackage{amssymb}

\makeatletter

\theoremstyle{plain}
\newtheorem{thm}{\protect\theoremname}[section]
\theoremstyle{definition}
\newtheorem{defn}[thm]{\protect\definitionname}
\theoremstyle{plain}
\newtheorem{prop}[thm]{\protect\propositionname}
\theoremstyle{plain}
\newtheorem{lem}[thm]{\protect\lemmaname}
\theoremstyle{plain}
\newtheorem{cor}[thm]{\protect\corollaryname}
\theoremstyle{definition}
\newtheorem{example}[thm]{\protect\examplename}
\theoremstyle{remark}
\newtheorem{rem}[thm]{\protect\remarkname}
\theoremstyle{plain}
\newtheorem{conjecture}[thm]{\protect\conjecturename}

\makeatother

\providecommand{\conjecturename}{Conjecture}
\providecommand{\corollaryname}{Corollary}
\providecommand{\definitionname}{Definition}
\providecommand{\examplename}{Example}
\providecommand{\lemmaname}{Lemma}
\providecommand{\propositionname}{Proposition}
\providecommand{\remarkname}{Remark}
\providecommand{\theoremname}{Theorem}

\begin{document}
\title{Title}
\maketitle
\begin{abstract}
...
\end{abstract}

\section{Introduction}

\section{Polyhedral....}

Let $\mathcal{K}$ be a finite set of indices.
\begin{defn}
	For each $k\in\mathcal{K}$, let $\widehat{\phi}_{k}:\mathbb{R}^{3}\times\mathbb{R}\rightarrow\mathbb{R}$
	be the function defined by
	\[
	\widehat{\phi}_{k}\left(x,t\right)\coloneqq x\cdot n_{k}\left(t\right)-a_{k}\left(t\right),
	\]
	where $n_{k}:\mathbb{R}\rightarrow S^{2}$ and $a_{k}:\mathbb{R}\rightarrow\mathbb{R}$
	are $C^{\infty}$ functions. For each $k\in\mathcal{K}$ and $t\in\mathbb{R}$,
	let $\omega_{k}\left(t\right)$ be the affine set defined by
	\[
	\omega_{k}\left(t\right)\coloneqq\left\{ x\in\mathbb{R}^{3}\mid\widehat{\phi}_{k}\left(x,t\right)<0\right\} .
	\]
	For each $t\in\mathbb{R}$, let $\Omega_{t}$ be the convex set of
	$\mathbb{R}^{3}$ defined by
	\[
	\Omega_{t}\coloneqq\bigcap_{k\in\mathcal{K}}\omega_{k}\left(t\right).
	\]
\end{defn}


\section{Numerical tests}

\section{Conclusions}

\bibliographystyle{plain}
\bibliography{references}


\section*{Appendix}
\end{document}
